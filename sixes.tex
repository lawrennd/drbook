%{
\begin{octave}
  %}
  % Comment/MATLAB set up code
  importTool('dimred')
  dimredToolboxes
  randn('seed', 1e6)
  rand('seed', 1e6)
  if ~isoctave
    colordef white
  end
  colorFigs = false;
  % Text width in cm.
  textWidth = 13
%{
%Start of Comment/MATLAB brackets
\end{octave}
In \refchap{chap:thinking} we discussed the use of
mixtures of Gaussian densities for modelling data. The motivation was
that the center of each Gaussian density represented a ``prototype''
data point that was then corrupted through various
transformations. The corruption of the points was modeled by Gaussian
densities.  Here we would like to consider a more realistic corruption
of a prototype.

Let's consider an image of the handwritten digit 6 (image
br1561\_6.3.pgm), taken from the \fixme{USPS ??} data. The image has
$64$ rows by $57$ columns, giving a data point which is living in a
$\dataDim=3,648$ dimensional space. Let's consider a simple model for
handwritten 6s. We model each pixel independently with a binomial
distribution\book{ (\refbox{box:multinomial}).}{.} Each pixel is the result of a
single binomial trial.
\[
p(\dataVector) = \prod_{j=1}^\dataDim \binomProb^{\dataScalar_j}
(1-\binomProb)^{(1-\dataScalar_j)},
\]
where the probability across pixels are taken to be independent and
are governed by the same probability of being on, $\binomProb$. The
maximum likelihood solution for this parameter is given by the number
of pixels that are on in the data point divided by the total number of
pixels\book{ (see \refbox{box:binomial}):}{:}
\[
\binomProb = \frac{849}{3,648}.
\]
The probability of recovering the original six in any given sample is
then given by\footnote{Note this is not the binomial distribution
  which gives probabilities for total number of successes in a series
  of binary trials, here we don't only need a precise number of
  successes, we also need the successes to occur at the right pixels.}
\fixme{Add number of ones}
\[
p(\dataVector=\text{given\,image}) = \binomProb^{849}(1-\binomProb)^{3,648 - 849
}=2.67\times10^{-860}.
\]
\begin{boxfloat}
  \caption{The Binomial Distribution}\label{box:binomial}

  \boxfontsize
  The binomial distribution is the special case of the multinomial where
  the outcome of the trials is binary (success or failure). It
  represents the distribution for a given number of successes in a set
  number of trials when the probability of success is constant. For a
  probability of success given by $\binomProb$ the binomial distribution
  has the form
  \[
  p(n|N, \binomProb)=\frac{N!}{n!(N-n)!}  \binomProb^n (1-\binomProb)^{N-n}
  \]
  where $n$ is the number of successes and $N$ is the total number of
  trials. The term $\frac{N!}{n!(N-n)!}$ is known as the binomial
  coefficient and represents the number of ways in which $n$ successes
  can occur in $N$ trials (for example 1 success in 2 trials can occur
  either by success then failure or by failure then success. That's two
  ways: $\frac{2!}{(1!\times (2-1)!)} = 2$).

  A special case of the binomial is the when $N=1$. Then it is sometimes
  known as the Bernoulli distribution,
  \[
  p(s|\binomProb) = \binomProb^s (1-\binomProb)^{1-s}.
  \]
  For a model of binary pixels in an image that assumes each pixel is
  independent we have
  \[
  p(\dataVector|\binomProb) = \prod_{j=1}^{\dataDim}
  \binomProb^{\dataScalar_j} (1-\binomProb)^{1-\dataScalar_j}
  \]
  where $\dataVector$ is a vector of binary pixel values from the image
  and the image has $\dataDim$ pixels. The negative log likelihood is
  \[
  \errorFunction(\binomProb) = -\sum_{j=1}^\dataDim\dataScalar_j \log \binomProb -
  \sum_{j=1}^{\dataDim}(1-\dataScalar_j) \log(1-\binomProb)
  \]
  and gradients with respect to $\binomProb$ can be computed as
  \[
  \diff{\errorFunction(\binomProb)}{\binomProb} =
  \frac{1}{\binomProb}\sum_{j=1}^\dataDim \dataScalar_j -
  \frac{1}{1-\binomProb}\sum_{j=1}^\dataDim (1-\dataScalar_j)
  \]
  which can be re-expressed using the fact that the number of pixels
  that are on is $\dataDim_1=\sum_{j=1}^\dataDim\dataScalar_j$ and the
  number of off pixels is $\dataDim -\dataDim_1 = \sum_{j=1}^\dataDim
  (1-\dataScalar_j)$,
  \[
  \diff{\errorFunction(\binomProb)}{\binomProb} = \frac{\dataDim_1}{\binomProb} -
  \frac{\dataDim-\dataDim_1}{1-\binomProb}
  \]
  setting to zero and rearranging we recover
  \[
  \binomProb = \frac{\dataDim_1}{\dataDim-\dataDim_1}{1-\binomProb}
  \]
  at a stationary point. Further rearrangement leads to
  \[
  \binomProb =\frac{\frac{\dataDim_1}{\dataDim-\dataDim_1}}{\left(1 +
      \frac{\dataDim_1}{\dataDim-\dataDim_1}\right)} =
  \frac{\dataDim_1}{\dataDim}
  \]
  which turns out to be a global minimum of the negative log
  likelihood. To see this we simply compute the second derivative of the
  negative log likelihood with respect to $\binomProb$,
  \[
  \diffTwo{\errorFunction(\binomProb)}{\binomProb} = -
  \frac{\dataDim_1}{\binomProb^2} -
  \frac{\dataDim-\dataDim_1}{(1-\binomProb)^2},
  \]
  which will always be negative if $0 \leq \binomProb \leq 1$ as it must
  be. This implies the negative log likelihood is convex and the
  stationary point is a global minimum.
\end{boxfloat}
This implies that even if we sampled from this model every nanosecond
between now and the end of the universe\footnote{We follow
  \cite{MacKay:information03} in assuming that will be in
  approximately \fixme{$number$} years.} we would still be highly
unlikely to see the original six. Our expected waiting time would be
$10^{217}$ years and there is only a \fixme{$fill$} probability of seeing it
before the universe ends\footnote{This can be computed with the geometric distribution.}. This is clearly a very poor generative
model of the six. The independence assumption across features (just as
in \refchap{chap:thinking}) is a very poor one. Let's consider an
alternative model for handwritten sixes. We generate a data set in the
following way: we take the prototype 6 and rotate the image 360 times,
by one degree for reach rotation. This corruption of the prototype 6
is designed to reflect the sort of corruptions that real prototypes
might undergo. A rotated 6 is still a 6, and although it is not
realistic to consider such drastic rotations as we have applied, it
will be helpful in visualizing the corruption of the data. In
\reffig{fig:prototypeSix} we show the original digit and rotations of
10 degrees clockwise and anticlockwise.

\begin{figure}
  \begin{octave}
    %end of Comment/MATLAB brackets.
    %} 
    close all
    % Load in the six.
    sixImage = dimredLoadSix;
    rows = size(sixImage, 1);
    col = size(sixImage, 2);


    figure(1)
    imagesc(sixImage);
    axis off
    axis equal
    if ~colorFigs, colormap gray; end
    printLatexPlot('demSix0', '../../../dimred/tex/diagrams/', 0.33*textWidth);
    for i = 1:3
      randImage = rand(rows, col)>length(find(sixImage))/(rows*col);
      clf
      imagesc(-randImage);
      axis off
      axis equal
      if ~colorFigs, colormap gray; end
      printLatexPlot(['demSixSpace' num2str(i)], '../../../dimred/tex/diagrams/', 0.33*textWidth);
    end

    options.noiseAmplitude = 0;
    options.subtractMean = false;

    Y = generateManifoldData('six', options);

    ind = [31 11 350 330 310 290];
    for i = 1:length(ind)
      clf
      imagesc(-reshape(Y(ind(i), :), 64, 64))
      if ~colorFigs, colormap gray; end
      axis off
      axis equal
      printLatexPlot(['demSix' num2str(i)], '../../../dimred/tex/diagrams', 0.33*textWidth);
    end

    %{
  \end{octave}
  \begin{center}
    \subfigure[Original 6]{\input{../../../dimred/tex/diagrams/demSix0}}\hfill
    \subfigure[Rotation through 10 degrees clockwise]{\input{../../../dimred/tex/diagrams/demSix3}}\hfill
    \subfigure[Rotation through 10 degrees anticlockwise]{\input{../../../dimred/tex/diagrams/demSix2}}
  \end{center}
  \caption{The prototype 6 taken from the \fixme{USPS} data along with
    two example rotations of the digit. The original image has been
    converted to 64$\times$64 to match dimensionality of the rotated
    images.}\label{fig:prototypeSix}
\end{figure}

In \reffig{fig:rotatedSixes} we take all 360 examples of the rotated
digits and visualize these data using principal component analysis.
and linearly project them down to a two dimensional space (using the
posterior given in \refsec{sec:ppcaPosterior}). Selected examples from
the data are also placed on the plot near their corresponding
projected position. What we see makes intuitive sense. The data,
projected into the two dimensional space, proscribes a circle. The
data set is inherently one dimensional. The dimension of the data is
associated with the rotation transformation. A pure rotation would
lead to a pure circle. In practice rotation of images requires some
interpolation and this leads to small corruptions of the latent
projections away from the circle.
\begin{figure}
  \begin{octave}
    %end of Comment/MATLAB brackets.
    %} 
    % close all
    demSixDistances(4096)
    printLatexPlot('demSixDistances4096', '../../../dimred/tex/diagrams', 0.9*textWidth);
    %{
  \end{octave}
  \begin{center}
    \input{../../../dimred/tex/diagrams/demSixDistances4096}
  \end{center}
  % \fixme{Another plot that is a zoom in on some portion of the plot??}
  \caption{Inter-point squared distances for the rotated digits data. Much of the
    data structure can be seen in the matrix. All points are ordered by
    angle of rotation. We can see that the distance between two points
    with similar angle of rotation is small (note in the upper right and
    lower left corners the low distances associated with 6s rotated by
    roughly 360 degrees and unrotated 6s.}\label{fig:sixDistances4096}
\end{figure}
%}
%%% Local Variables: 
%%% mode: latex
%%% TeX-master: "mds"
%%% End: 
